\section{Discussion and Conclusion}
\label{sec:conclusion}

Our empirical analysis of Namecoin and exploration of the design space reveal several potential mechanism design weaknesses. First, Namecoin's fees are far too low to deter squatters. The system utilizes neither an auction nor algorithmic pricing that might help price names near their market value. Worse, users who paid a high network fee in the early period and who wish to leave the system have no way to recoup any of their investment by returning the name to the primary market.

These problems are exacerbated by the lack of a functioning secondary market. Buying a squatted name, even if contact information is available on the block chain, is a cumbersome manual process. There is no way to, say, search for all names available for sale matching a given string and below a specified price limit. Namecoin does not integrate an exchange into its core functionality, nor have Namecoin developers chosen to build and promote one outside the system.

On the other hand, the online identity service, OneName, has a very large user base. We found that after limiting the number of name/value pairs to only those with unique, well formed entries (using the strategies listed in section~\ref{sec:methods}) OneName has roughly 20,000 pairs, which dwarfs the 1000 or so .bit domains. It is too early to tell whether or not these users are putting their OneName identities to any real use, but at least in terms of registration numbers interest appears to be high.

Perhaps the biggest reason for this is that while scarce, OneName names are not nearly as valuable as domain names, and so it is far less important to get the mechanism design right, or have any mechanism at all. Unlike domain names that must be remembered by humans, users typically find OneName identities through links from other social media services or by searching for the individual's real name.

A second reason is that OneName benefits from being a centralized service on top of a decentralized platform. If a name is squatted, a typical user who wants that name might not even know how to contact the squatter because they are trying to register a name through the OneName website which doesn't facilitate this process. While a user could technically interact with the `u/' subspace of the block chain directly, most OneName users don't.

The .bit subspace, on the other hand, suffers from the lack of any centralized service providers coupled with the poor usability of the client software. A new user faces many significant hurdles in registering a name. The first step is acquiring a Namecoin client which requires downloading, verifying, and storing the entire block chain, which is currently about 1.6 GB. Then she must acquire NMC, which requires connecting the user's wallet on a cryptocurrency exchange (which supports Namecoin) to her bank account. In our experience, there is a 3-day wait after registering with an exchange and being allowed to transfer funds. Now the user is finally ready to register names, but she must also remember to update names every 250 days or run software in the background that will automatically handle this.

In conclusion, designers of decentralized namespaces must pay careful attention to mechanism design to deter squatters and facilitate a healthy secondary market. Further, the usability of the overall ecosystem, which includes not just users who own names but the applications that utilize the namespace for web browsing or other tasks, is paramount. Finally, a hybrid model of centralized services that utilize an underlying decentralized platform is worth considering.

%
%As mentioned in (Section \ref{sec:background}), OneName completely abstracts away the involvement of Namecoin for its users, paying all of their fees. One would think that because OneName is free, it should suffer from a massive number of squatters. According to the data in the blockchain, however, this is simply not true. OneName actually has a much better user to squatter ratio than {\tt d/} with only a few thousand squatters on on their namespace (including OneName itself which has benevolently squatted some of the names in its subspace to save usernames for well known individuals). This is a result of a few different factors. The first factor contributing to this is that in order to make the name/value pairs for free, an individual needs to go through OneName. This allows OneName to implement their own techniques to try to prevent a single user from claiming many usernames. Additionally, because a user goes through OneName, OneName actually owns the Namecoin address that holds the token with their data. If a squatter was to squat names purchased through the OneName website, they would be unable to sell them to a buyer because OneName owns the name. Even if a squatter isn't intimidated by the small fee and they try to squat through the Namecoin client, squatting OneName usernames is less valuable than squatting domains in the .bit subspace. A certain name in the OneName namespace is not particularly valuable because a user doesn't really need a specific name. If John Smith is using OneName, {\tt jsmith} would be a pretty ideal username, but he could just as easily use {\tt jsmith1} or {\tt jsmith912}. With any of these usernames, when someone searches "John Smith" in OneName, his identity will come up because OneName stores an individual's actual name in the value portion of the name/value pair, and this is how most people will search for them (unlike a domain name). Finally, a squatter would be hard pressed to find buyers in this namespace. For domain names, if a user sees that a name they want is taken, they can look at the value stored on the blockchain, see that it has a squatters contact information, and contact the squatter. As discussed previously, OneName users are insulated from the Namecoin network. They would be trying to make a new account on the OneName website which would simply tell them that a particular username is taken. Even if they are willing to pay for it, they wouldn't know who to contact without having to directly engage with Namecoin themselves. 
%
%Another final interesting fact derived from OneName's success indicates how users value the distributed cryptographic security of their information compared to the simplicity of a service. Because OneName owns all of the name/value pairs, and the users typically never validate what OneName tells them with the blockchain, the users are completely trusting it as a central authority which leaves them vulnerable. Most users find this acceptable because even though they aren't checking the Namecoin blockchain, they know that someone could be checking the Namecoin blockchain. As long as there is the potential for someone to verify the information with the blockchain, OneName can't lie without being caught. Judging by the success of OneName, users find this pseudo security to be a close enough approximation to that provided by the Namecoin blockchain if it allows them to ignore having to deal with the Namecoin protocol.
%
%
