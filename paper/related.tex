\section{Related Work}
\label{sec:related}

The ability to publish messages to the block chain immediately allows a variety of applications. Physical property, shares of stocks, bonds, or any other real asset can be digitally represented on the block chain. Overlay protocols such as Mastercoin and Colored Coins specify a syntax and semantics for such digital representations \cite{mastercoinspec, rosenfeld2012overview}. CommitCoin allows for putting hash commitments on the Bitcoin block chain in order to timestamp data in a trustless manner \cite{clark2012commitcoin}. 

There are a number of more complex applications that require additional primitives. Financial derivatives are contracts whose value depends, in some mutually agreed-upon way, on the price (or movements in price) of an underlying asset. Implementing derivatives of digital assets, then, requires user-defined logic (or scripts) for transaction validation. This can be accomplished via an altcoin with a flexible scripting language such as Ethereum \cite{ethereumwhitepaper}. Furthermore, since derivatives depend on prices, scripts that implement them require access to price feeds as input. Otherwise, it requires a trusted third party (known by a Bitcoin address) to regularly publish suitably encoded, signed price statements reflecting external reality. These can be published directly to the block chain or distributed off-chain and only added to the blockchain when needed to redeem an asset. More generally, such entities could publish any data feed representing news or other events.

Bitcoin can act as a platform for fair secure multiparty computation \cite{andrychowicz2014secure, bentov2014use, kumaresan2014use}. These are SMC protocols augmented with Bitcoin operations, e.g., a payment from one participant to another, or a deposit.

Finally, a set of related ideas known as decentralized autonomous organizations (among several other names) has stirred considerable excitement in the community recently. These combine several primitives discussed above --- digital assets, long-lived scripts implementing arbitrary logic governing those assets, data feeds, and out-of-band communication. Some proposals for DAOs incorporate human input in various forms: one is a decentralized agent farming out computationally intractable tasks to humans \cite{buterindao}. Another is voting by shareholders of decentralized agents to enable modifications to the logic (i.e., script). As with any input external to the block chain, these must be implemented using data feeds.

% The block chain has been proposed as a stratum for building technologies and mechanisims beyond storing financial transactions. Hash commitments appear in the Namecoin protocol for registering new names. In  \cite{bonneau2014decentralizing} the authors describe how to use the block chain to create prediction markets and order books.