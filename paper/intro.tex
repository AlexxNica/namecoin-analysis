\section{Introduction}
\label{sec:intro}

We initiate the study of {\em decentralized namespaces} from an economic perspective. A namespace, as we define it, is an online system that maps names to values. The Domain Name System (DNS) is the most prominent example.  A web service such as Twitter that allows users to claim usernames and create profiles can be thought of as implementing a namespace. To be memorable by humans, namespaces must support arbitrary user-chosen strings as names. To be secure, namespaces must map each name to the same value for all users and adversaries shouldn't be able to convince a user that any other value is correct.

The problem of {\em decentralized} namespaces has long been recognized as an important one. The DNS is a critical yet centralized component of the Internet and those who control it can alter the web for all users. The controversy around the 2011 domain name seizures by the U.S. DoJ and DHS illustrates why many researchers and activists have sought decentralized alternatives \cite{}.

The above three properties --- security, user-chosen names, and decentralization --- are known as {\em Zooko's triangle}. Until 2011, designing a system that exhibited all three was conjectured to be {\em impossible}. The rationale was that enforcing the uniqueness of name-value mappings and a consistent view of the directory for all participants would require a centralized server or a hierarchy.

Cryptocurrency technology enables building a namespace with all three properties. Put simply, the block chain is a global, distributed data structure that can be repurposed as a directory. {\em Miners} execute a consensus protocol to establish the state of the system and are incentivized to do so by mining rewards they receive in exchange for their participation. As long as a majority of miners --- weighted by computing power --- follow the protocol, all users will see a consistent view of the directory when they query it. This in turn gives the system, and hence the underlying currency, economic value, making miners' actions profitable to them.

{\em Namecoin} is a cryptocurrency that realizes a decentralized namespace. [Basic details: altcoin, first fork of bitcoin, updates, transfers, different types of fees, different applications --- .bit and OneName.]

Namecoin offers a novel solution to the technical challenges of decentralized namespaces. However, there are also economic challenges. These arise from the fact that even though namespaces theoretically support an infinite number of names, the supply of names that are memorable and meaningful to humans is scarce. Allocating these names to users is therefore a mechanism-design challenge. {\em The central thesis of this work is that this mechanism design challenge is far harder than realized.} A system that gets it wrong may ``work" in a narrow technical sense, but may not be useful to real users.

Specifically, there are several crucial questions to consider: how do we model the economic behavior of the users of a namespace and what are the goals of mechanism design for namespaces? How well does Namecoin succeed at attaining these goals and what are its limitations? If Namecoin is not the ideal design, can we analyze the design space as a guide to the creators of future decentralized namespaces?

Along the above lines, we make the following contributions. We begin by proposing a model of utility of different names to different participants and articulating desiderata of a decentralized namespace in terms of this utility function (Section \ref{sec:model}). \hi{[Elaborate]}.

The central contribution of this paper is a thorough empirical analysis of the Namecoin ecosystem. We develop a series of criteria based on the block chain, network behavior, as well as content to distinguish active websites from parked or squatted names (Section \ref{sec:squatting}). This allows us to iteratively filter our dataset of around 140,000 registered names in Namecoin, leaving a mere \hi{N} active websites.

We then delve deeper into the economics of names. Namecoin has a built-in ability to transfer names, which is a secure way to trade them using the block chain. However, it is not obvious which transactions correspond to such sales, as opposed to regular name updates. We develop a novel analytic technique to distinguish the two types of transactions and find evidence for at most \hi{M} transactions in the entire history of Namecoin that may represent sales of names (Section \ref{sec:}). Of course, it is possible that more sales have occurred off the block chain, but there doesn't appear to be a widely known marketplace for Namecoin names.

Next, we develop a model for estimating the value of a name based on characteristics such as its length, the frequency of the name (treated as a word) on web pages, the Alexa rank of the corresponding .com, and several other factors (Section \ref{sec:}). In addition to the obvious goal of algorithmically pricing names, we seek to understand what makes a name valuable in the Namecoin community. Since there is almost no price data on sales of names, we instead use the observations we can make on the block chain about how names are treated by their owners. We posit that several characteristics such as a name being registered early in Namecoin's history and being re-registered quickly after expiry are indicative of more valuable names. We find that \hi{[enumerate characterstics]} are correlated with markers of high value. The high level of variation we find in the estimated value of different names further underscores the scarcity of names and the need for careful mechanism design.

Based on all the empirical evidence we present, we are left to conclude that the Namecoin ecosystem is dysfunctional \hi{Here and throughout the text, avoid confounding Namecoin and .bit}. The vast majority of registered names represent squatting and there is little evidence of a secondary market for names. While there could be many factors that explain the lack of adoption, there appears to be clear room for improvements in the design to minimize squatting and other problems. To this end, in Section \ref{sec:}, we explore the design space of decentralized namespaces and make recommendations.

While decentralized namespaces have many important applications, their promise hasn't been realized so far. Our work helps understand why this might be and lays the groundwork for a more rigorous approach to building such systems.

