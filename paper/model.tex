\section{Model}
\label{sec:model}

{\bf Names are scarce.} Many types of systems map keys or names to values. In his essay introducing his triangle, Zooko considers naming systems that include a system mapping PGP key fingerprints to keys. Fingerprints are hashes of keys. This would not qualify as a namespace by our definition because users cannot pick arbitrary names --- that would require finding the hash pre-image of an arbitrary name (fingerprint), a hard problem by the definition of a hash function. Zooko (and similar prior work) considers such systems to lack human memorability of names. We replace this criterion with that of user choice of names, which is roughly equivalent but easy to define rigorously. 

In a system that maps key fingerprints to keys, names are fungible and essentially infinite, and thus not scarce and have no market value. On the other hand, in any system that we call a namespace, scarcity is an almost inevitable consequence of free user choice. Even when names are scarce, the system architecture can make names more or less valuable. For example, usernames on Twitter are more valuable than on Facebook --- the username or handle is a relatively important way to find a user on Twitter, whereas on Facebook it is done more often by navigating the social graph.

The scarcity of names makes the design of namespaces challenging. But it also makes the problem particularly amenable to cryptocurrency-based solutions. Since names have market value, and participation in the primary market requires the associated cryptocurrency, the currency becomes valuable. This incentivizes miners, making the system secure.

{\bf The (non) role of trademarks.} Scarcity is also an issue in centralized namespaces, but it manifests in different ways. One key goal of most centralized systems is trademark protection. The trademark system can be seen as a legal solution to the problems of name scarcity and limits of human memorability in the real world, independent of any particular technological system. Most centralized systems support some form of arbitration or dispute resolution in which trademark plays a role.\footnote{For example see ICANN's Uniform Domain-Name Dispute-Resolution Policy \cite{walker2000icann}.} In a decentralized system, on the other hand, there is no easy way to enforce any rules that cannot be encoded algorithmically, and as yet there is no way to cryptographically assert that one is the owner of a trademark, and this may be impossible given the complexity and nuance of modern trademark law.

{\bf Primary and secondary markets, algorithmic agents.} We use the term primary market to mean the part of the market where new names are issued to users for the first time. By contrast, the secondary market deals with the sale of a name already in use. 

Whom do names ``belong to'' before they are sold on the primary market? To answer this, we must understand decentralized agents. Any cryptocurrency can be thought of as an algorithmic agent executed as a global, distributed computation. The agent has no capability to store private information but it can hold funds and transact with users of the system. Bitcoin encodes a very simple agent whose only function is to release currency into circulation at a pre-specified rate.\footnote{At the other extreme, altcoins such as Ethereum allow any user to create an agent by making the central agent ``Turing complete,'' that is, flexible enough to execute arbitrary programs specified by users on their behalf.} 

In a nutshell, the agent implemented by a namespace initially owns all names and sells them to users. As we'll see in Section \ref{sec:design}, there are many variants including whether names are sold or leased, whether or not they can be bought back, how names are priced, what the agent does with the funds received as payment, and so on.

The most straightforward implementation of a secondary market is to leave it entirely external to the system. This is the approach taken by Namecoin, but again variety of choices are possible. We elaborate in Section \ref{sec:design}.

{\bf Squatting.} Even though names are scarce, it's not obvious what ``squatting'' means. After all, material goods are scarce, but we don't usually characterize car owners as squatting on them. The difference is that names have (vastly) different utility to different users. A squatted name is one that is owned by a user whose utility for that name is close to zero (or very small compared to the user who has the most utility for that name), purchased in hopes of selling to another user whose utility is higher.

Is squatting a problem for the system? This is a tricky question. If squatting exists merely because one side of the primary market is an algorithm and doesn't charge market price, then squatters could be seen as analogous to brokers or ticket scalpers. Such agents are sometimes considered to perform a useful function as market makers \cite{silber1984marketmaker}, or at least tolerated, but not considered an existential threat to the functioning of the market.

On the other hand, squatters can be viewed as analogous to land speculators. This analogy is supported by the fact that names, like land but unlike tickets, are not fungible. The speculator may have zero utility for a name and makes no use of the name himself; he hopes that demand for a name will rise in the future, and may therefore not sell to a buyer who has a positive utility today. This could lead to a market failure in a few ways. The uncertainty around future demand means that some names may be squatted indefinitely, while legitimate users may end up with sub-optimal names. If most valuable names are locked up by squatters, it may prevent the growth of the sytem, in turn preventing the growth in market price for names that speculators are hoping for. Land speculation has also been criticized as contributing to a market failure \cite{ottensmann1977urban}. 


In this view, squatting may exist and may be problematic even if the primary market is able to discover the market price. But if the primary market is algorithmic, it only exacerbates the squatting problem.


{\bf Complexity of the utility function.} 
Our argument above is essentially that utility functions are time-varying. This is particularly relevant to cryptocurrency-based namespaces, since they need to ``bootstrap,'' and many users may have a zero or negligible utility for all names until they decide to participate in the namespace at all.

Another aspect of complexity of the utility function is diminishing marginal utility. A user named John Smith may want either the name \textsf{JohnSmith} or \textsf{john-smith} but not necessarily both. Time-variation and diminishing marginal utility make the mechanism design problem very tricky.

Suppose, to the contrary, that utility functions are time invariant, and utility functions of a user for different names are independent. Then we may state the goal of the system as ownership of each domain by the user with the maximum utility (or willingness to pay) for that domain, as long as that value is above some threshold. We can easily realize this by making the primary market a second-price auction with a reserve price. No secondary market would be necessary.

Of course, this is oversimplified. Let's add diminishing marginal utility to this model. Suppose there are 10 John Smiths in the world and 10 variations of the name \textsf{JohnSmith}. A 10x10 matrix describes the utility of each user for each name. Further, none of these users has any utility for more than one name. Now we can state a welfare-maximization goal of finding the assignment of names to users that maximizes the sum of each user's utility for his assigned name, which translates to a maximum weighted graph matching problem. Or we could be happy with a Pareto-efficient assignment; since this is one-sided market (names aren't agents), this is the house allocation problem \cite{abdulkadiroglu2013matching}. This shows that even in a highly simplified setting with a finite number of names and the same number of users, the mechanism design question is very tricky.

When we add time-varying preferences to the model, it is unclear if we can even formally state a goal for the system. The harder we make it for a user to hold on to a purchased name, the easier it becomes for an adversary to ``seize'' a name (see below).

At any rate, the fact that preferences are time varying appears to be part of the reason that names in Namecoin expire and must be renewed; in effect, they are leased rather than purchased outright.

{\bf Seizures.} The Namecoin community considers it a major goal to prevent ``seizures'' of domain names by adversaries \cite{censorship}. To prevent seizures, then, it must be easy to hold on to a name after buying it. This is in tension with preventing speculation, which works by buying a name and holding on to it despite other people wanting it.

The reason it is even technically possible to prevent such censorship in the face of a well-funded adversary is that the adversary can't possibly pre-emptively buy up all the names that the victim might use. For example, Wikileaks might find any name with the string ``wikileaks" in it to be acceptable, even if not ideal, for their purposes. In other words the victim's utility function has a large support; the adversary's utility for a name is {\em contingent on} the victim owning that name.


%It turns out that the utility of an adversary for a name can be predicated on the ownership of that name by a particular user. For example, a government may want wikileaks.bit to shut it down if Wikileaks owns it and is operating it, but otherwise may have a much smaller utility for that domain. (After all, the government couldn’t hope to pre-emptively buy up all the names that Wikileaks might use.) To prevent seizures, then, it must be easy to hold on to a name after buying it. This is in tension with preventing speculation, which works by buying a name and holding on to it despite other people wanting it.





